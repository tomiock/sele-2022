% !TeX spellcheck = ca
\documentclass[arial,a4paper,print]{article}

\usepackage{helvet}
\usepackage{float}
\usepackage{graphicx}
\usepackage{tipa}
\usepackage{subcaption}
\usepackage[labelfont=sc, font={footnotesize, singlespacing}]{caption}
\usepackage[margin=2cm]{geometry}  

\renewcommand{\familydefault}{\sfdefault}

\usepackage[catalan]{babel}
%opening
\title{Català: Selectivitat 2022}
\author{tomiock}

\begin{document}
\maketitle

\section{Preguntes concretes sobre gramàtica}

\subsection{Escollir entre dues paraules}
\begin{itemize}
\item Quan/Quant: 
\subitem quan: adverbi temporal
\subitem quant: adjectiu per indicar quantitat
\item Què/Que: 
\subitem què: 
\subitem que: 
\item Si no/Sinó:
\subitem si no: condicional + adverbi negació \textit{Si no vols això...}
\subitem sinó: \textit{No ho ha fet ell, sinó ella.}
\item Potser/Pot ser: 
\subitem pot ser: verb + ser
\subitem potser: adverbi que dona la possibilitat del que es diu \textit{pot ser vindrà}. Equival a \textit{tal vegada} o \textit{probablement}. Al substituir \textit{pot ser} per \textit{probablement} s'haurá de treure la conjunció que \textit{Pot ser que vingui} $\rightarrow$ \textit{Probablement vingui}
\item Gens/Res:
\subitem gens: equival a \textit{en absolut}. 
\subitem res: equival a \textit{cap cosa}. 
\item Gaire bé/Gairebé: 
\subitem gaire bé: equival a \textit{poc bé} en oracions negatives \textit{No em trobo gaire bé}  
\subitem gairebé: equival a \textit{quasi}: \textit{No érem quasi/gairebé tots}
\item Cadascú/Cadascun: 
\subitem cadascú: \textit{Cadascú/Cada u és lliure de...}
\subitem cadascun \textit{Li he donat a cadascun/a cada un...}
\item Tan/Tant: 
\subitem tan: davant de un adjectiu, adverbi o SPrep
\subitem tant: davant d'un nom
\item Incomplet/Incomplert: 
\subitem incomplet: no completat
\subitem incomplert: incumplido 
\item Fons/Fondo:
\subitem fons: nom invariable
\subitem fondo: adjectiu o adverbi \textit{respirar fondo} o \textit{un plat fondo}
\item Alhora/A l'hora: 
\subitem alhora: equival a \textit{a un mateix temps}
\subitem a l'hora: significa literalment a \textit{a aquesta hora}
\end{itemize}

\subsection{Conjugació de Verbs}
\begin{itemize}
\item Caber: 
\item Fondre: 
\item Concloure: 



\end{itemize}

\section{Fonètica}
\subsection{Grafies importants}
\begin{itemize}

\item \textipa{[z]}: 
\item \textipa{[s]}: 
\item \textipa{[tS]}: 
\item \textipa{[Z]} o \textipa{[dZ]}: 
\item Sons Africats: 
\item Sons Nasals: 
\item Sons Palatals: 

\end{itemize}

\section{Lectures Obligatòries}

\subsection{\textit{La Plaça del Diamant}}

\subsection{\textit{Aigües Encantades}}



	
\end{document}