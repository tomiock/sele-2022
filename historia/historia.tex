% !TeX spellcheck = ca
\documentclass[arial,a4paper,print]{article}

\usepackage{helvet}
\usepackage{float}
\usepackage{graphicx}
\usepackage{subcaption}
\usepackage{outlines}
\usepackage[labelfont=sc, font={footnotesize, singlespacing}]{caption}
\usepackage[margin=2cm]{geometry}

% indentation 
\setlength{\parindent}{1.5cm}	
\setlength{\parskip}{.2cm plus4mm minus3mm}
\renewcommand{\baselinestretch}{1}

\renewcommand{\familydefault}{\sfdefault}

\newcommand{\subItem}[1]{
	{\setlength\itemindent{15pt} \item[-] #1}
}

\usepackage[catalan]{babel}

\title{Historia: Selectivitat 2022}
\author{tomiock}

\begin{document}	
\maketitle

\section{La Restauració: evolució política, social, econòmica i demogràfica (1875-1931)}
Esto no se estudia. 

\section{El catalanisme polític: precedents, aparició i evolució (1833-1931)}
Esto mucho menos. 

\section{La Segona República (1931-1936)}

\section{La Guerra Civil (1936-1939)}

\subsection{Esclat de Guerra Civil a Catalunya}
Godet va ser el responsable de la insurrecció a la zona Catalunya-Balears. El aixecament inicial va ser a les Balears (Mallorca i Eivissa) però el comandament militar de Barcelona era lleial a la Rª, per tant va haver-hi una gran resistència per part de les forces lleials al govern i les organitzacions obreres. 

L'aixecament va provocar una crisi en la societat catalana, va haver-hi una resposta immediata contra els Nacionals. 

Església, burgesia, propietaris, catòlics i classes altes van ser objecte de persecució, alguns van haver-hi de fugir. 

Sota el pretext de la revolució es van cometre molts delictes violents. 

El dia 18 (la insurrecció va ser el 17) els de la CNT i UGT havien assaltat armeries per poder tindre material per poder combatir. Juntament amb la Guàrdia d'Assalt i la Guàrdia Civil als Nacionals durant tot el dia 19. Els enfrontaments van acabar els dies 21-22 i els Nacionals es van rendir en altres llocs de Catalunya. 

Va sorgir una xarxa de poder popular vertebrada pels \textbf{sindicats i partits d'esquerres}. Els ajuntaments es van substituir per \textbf{comitès locals}. 

Per iniciativa de CNT/FAI es van crear el \textbf{Comitè de Milícies Antifeixistes}, estructurat a través de petites \textbf{columnes} de milicians. També es va encarregar de l'odre públic, l'indústria i el treball. 

Era evident el paper de les organitzacions d'esquerra era decisiu. Els rebels contaven amb molt poc suport a Catalunya i el suport a la Rª de part dels obrers era molt forta. 

Al Agost del 1936 es va crear el \textbf{Consell d'Economia de Catalunya}. Va ser encarregat de dissenyar un pla socialista per organitzar l'economia, va promulgar el \textbf{decret de col·lectivitzacions}. També hi aplicarien altres mesures com:
\begin{itemize}
	\item Col·lectivitzacions
	\item Control de la banca
	\item Entitats de crèdit públic
	\item Regulació de salaris
	\item Municipalització del sòl urbà
\end{itemize}

Companys que seguia sent President de la Generalitat, va posar a \textit{Josep Tarradellas} al cap d'un govern d'unitat que incorporava a:
\begin{itemize}
	\item ERC*
	\item CNT*
	\item PSUC
	\item POUM
\end{itemize}
Es va dissoldre el Comitè de Milícies Antifeixistes, que van ser substituïts pels ajuntaments, centralització dels serveis policials, aquesta mesura juntament amb una reconstrucció del sistema judicial van ser posades amb la fi de aturar la violència. També es van militaritzar les milícies, que es van enviar al Front d'Aragó. Finalment la Generalitat va iniciar una intervenció econòmica per complir amb les necessitats de la guerra.

\subsection{Fets de Maig}
Hi havien diversos punts de vista sobre el que fer. Els anarquistes volien millorar y \textbf{avançar socialment en el moment y també centrar-se en la guerra}. Però els Socialistes/Marxistes volien primer \textbf{centrar-se en la guerra} y després implementar les seves mesures polítiques. 
\underline{DESACORD EN LA ESTRATÈGIA DE GUERRA}

Aquest desacord va a anar fent-se més violent fins que al Maig del 1937 va acabar d'esclatar definitivament. Forces de la \textbf{Generalitat van desallotjar als anarquistes} que s'havien instal·lat en l'edifici de \textbf{Telefònica de pl.Cat}. 

Això va provocar \textbf{l'enfrontament entre CNT/POUM amb PSUC/ERC/UGT que donaven suport a la Generalitat}. La lluita era per tota la ciutat, fins y tot amb barricades. Va durar gairebé una setmana. Van morir un total de 200 persones. Y els anarquistes van ser derrotats. 

El Gobierno va intervenir a Barcelona amb 5000 guàrdies d'assalt per posar fi a la crisi. Això \textbf{va provocar tensions entre el Govern català i el Gobierno} (qüestionar la capacitat de la Generalitat per poder lluitar en una guerra i controlar els sectors polítics i sindicals més radicals). 

\subsection{Final de la Guerra a Catalunya}
Al juny es va formar un nou Govern de la Generalitat amb Companys com president (aliança entre ERC i PSUC, sense la CNT). Això va impedir que es tensessin les relacions amb el Gobierno. 

Després de la batalla del Ebre perduda pels republicans, aquests es van anar \textbf{retirant cap al nord}. Llavors les forces de Franco van anar avançant fins a ocupar el sud de Tarragona.  

Aleshores va començar la ofensiva final contra Catalunya i \textbf{es va conquistar tota Tarragona i Barcelona sense gaire resistència}. Al conquistar Girona la crisi del \textbf{exiliats} va començar (entre ells Azaña, Negrín i Companys). A començaments de Febrer tota Catalunya ja va ser conquistada. 

\subsection{Etapes militar/Batalles}
\subsubsection{Batalla Madrid}
Els Nacionals primer de tot volien Madrid, així que van anar avançant a través de \textbf{Badajoz y Toledo} amb les tropes de africanes de Franco (pont aeri). Al octubre ja eren a les portes de Madrid. 

Degut a la \textbf{mobilització general} que va decretar la Rª Madrid va resistir molt més del esperat. \textit{No pasarán, Madrid será la tumba del fascismo}. 

El Gobierno al traslladar-se a València, va deixar Madrid en mans del general Miaja i la resistència en mans del comandant Rojo. Es va resistir molt de temps principalment per les \textbf{brigades internacionals} i la \textbf{Columna Llibertat} (anarcosindicalistes que venien de Barcelona).  

Al organitzar-se l'exércit millor per defensar la ciutat es va \textbf{acabar la fase de columnes}. 

\subsubsection{Batalla Jarama i Guadalajara}
Es van iniciar dues maniobres per envoltar Madrid que van provocar la batalla de Jarama guanyada pels Nacionals. 

Després va passar la de Guadalajara que va ser una gran victòria per la Rª. 

\subsubsection{Batalla del Nord}
Ja que no es podia conquistar Madrid tan fàcilment, Franco i Mola es van centrar en Cantàbria i el País Basc. 

En aquest moment es quan es va produir els bombardeigs de les poblacions de Guernica per part de la aviació alemanya i italiana. 

\subsubsection{Fractura del Territori}
Al desembre del 1937 els republicans es van reorganitzar amb les \textbf{Brigades Mixtes} dirigides per Vicente Rojo (defensor de Madrid). Es volia un exèrcit més eficient i organitzat, per aquesta raó es van donar comandaments professionals. 

Es van començar a llançar ofensives republicanes com la \textbf{exitosa Batalla de Terol}. Però els Nacionals van començar \textbf{la campanya d'Aragó}, que va aconseguir que arribessin fins al mar, partint la Rº en dos.  

\subsubsection{Batalla del Ebre}
Llavors els republicans van llençar la \textbf{gran ofensiva del Ebre} des del nord, en un últim intent per salvar la Rª, però després d'avaçar durant un dies van haver-se de retirat i Catalunya va ser ocupada ràpidament sense gaire resistència. 

Va començar la crisi dels exiliats cap a França amb la caiguda de Tarragona, Barcelona i Girona; respectivament. 

\subsubsection{Fi de la guerra}
A partir d'aquí no va haver-hi cap batalla important degut a les poques forces que lis quedaven als republicans. Negrín va tornar a València després del exili a França per poder intentar resistir amb els Comunistes. Al cap d'uns dies \textbf{França i UK van reconèixer el Gobierno de Franco}. 

Llavors es va inicial una \textbf{petita insurrecció contra la Rª a Madrid pel coronel Casado}. Va intentar negociar amb Franco amb la recent creada \textbf{Junta de Defensa} però no es va acceptar cap condició proposta. Les tropes van entrar a Madrid sense cap resistència. 

Un cap Madrid estava en mans nacionals l'exércit va aconseguir acabar de conquistar València. Franco va signar a Burgos l'últim \textit{parte de guerra}, en el qual anunciava el final del conflicte. 

\subsection{Política Republicana Espanyola}
Insurrecció: Franco a Marroc (millor exèrcit), Mola a Pamplona, Queipo de Llano a Sevilla, Goded a Mallorca i Cabanellas a Saragossa. 

Al iniciar-se la insurecció el Gobierno va tardar en actuar, llavors Casares Quiroga va dimitir i Azaña va posar a \textbf{José Giral} va començar a \textbf{lliurar armes als sindicats} i els \textbf{partits del Front Popular}. Amb exèrcit de la Rª i la Guàrdia d'Assalt es va plantar cara als Nacionals, sobretot a Barcelona i Madrid (impedir aixecament a les grans ciutats). 

El Gobierno va decretar la dissolució del exèrcit tradicional i es van \textbf{crear batallons de voluntaris} que es van integrar en les  \textbf{milícies popular}. Van sorgir molts organismes revolucionaris a Astúries, Aragó, València i Madrid. 

Al empitjorar la situació contra els Nacionals es va posar en qüestió la dissolució del exèrcit. 

\paragraph{Gobierno de Largo Caballero}
El 5 de setembre, el socialista Largo Caballero va formar un \textbf{govern d'unitat} amb republicans, socialistes i al cap d'uns mesos 4 ministres anarquistes. El Gobierno es va traslladar a València. 

Caballero volia crear una gran aliança entre les forces republicanes, burgeses i obreres per poder guanyar la guerra. Ho volia fer a través de la reorganització del Estat, la militarització de les milícies i la \textbf{creació del Exèrcit Popular}. Però els desacords entre comunistes i anarquistes van donar problemes degut a que els partits s'allunyaven entre ells. Fins i tot els anarquistes en el Gobierno es resistien a integrar les seves milícies en l'exèrcit. 

\paragraph{Gobierno de Negrín}
Els Fets de Maig van disminuir l'influència anarquista i va \textbf{donar força als comunistes}, sobretot per l'ayuda de la URSS, que a canvi d'ajudes volien la dissolució del POUM (troskistes), davant l'insistencia del comunistes, Caballero va dimitir. 

El nou govern de socialista Negrín \textbf{va deixar d'inclure els sindicalistes} (UGT i CNT) i va il·legalitzar el POUM i els seus líders van ser detinguts. 

Tenia la prioritat de guanyar la guerra, llavors es va enfortir el poder central, amb una única direcció bèl·lica, \textbf{totes les milícies es van integrar en l'Exèrcit Popular}.  

Al novembre \textbf{es va traslladar el Gobierno a Barcelona} on també estava el Basc, per tal de controlar los zones amb més recursos que quedaven sota control republicà. El Gobierno va assumir molts poders directament, lo qual va crear conflicte amb la Generalitat. 

Negrín va proposar una \textbf{política de resistència de la Rª}, ja que no hi podia haver-hi cap acord amb els Nacionals que salvés la Rª. Franco no va acceptar cap mena de negociacions. 

Una vegada al març del 1938, quan ja gairebé no quedaven provisions, Negrín insistia en \textbf{la necessitat de resistir} per tal de esperar a que inici un conflicte a gran escala en Europa i poder salvar la Rª amb ajuda exterior. Es van redactar els 13 Punts de Negrín sense cap efecte (demanar una pau entre els dos bàndols). 

Un cop es va perdre Catalunya i el País Basc, la Rª tenia els dies contats. 


\subsection{Fonts}
\subsubsection{Situació Internacional}
\subsubsection{Situació de les Dones (dos bàndols)}
\subsubsection{Principis Ideològics dels NACIONALS}
\begin{itemize}
	\item Construcció Estat Autoritari
	\item Eliminació dels partits/sindicats d'esquerres
	\item Supressió de la Constitució
	\item Supressió Autonomia
	\item Repressió Institucionalitzada
	\item Estat Confessional
	\item Supressió Reformes Rª
\end{itemize}


Crònologia d'acciones polítiques:
\begin{itemize}
\item juliol 1936: Creació de la Junta de Defensa Nacional, composta per militars
\item octubre 1936: Franco es cap del gobierno i \textit{generalísimo} dels 3 exèrcits
\item abril 1937: Decret d'Unificació. Creació de la JONS i Falange com partits únics.
\item gener 1938: Formació del primer govern de Franco
\end{itemize}

\section{El Franquisme (1939-1975)}
\subsection{Inici/Definicions}
El nou estat franquista (Franco no tenia l'intenció de fer una dictadura transitòria) presentava aquestes caracteristiques:
\begin{itemize}
\item \textbf{Concentració de Poders}: Franco era el cap de l'estat, president del Gobierno i generalisimo de los tres ejercitos. 

\item \textbf{Totalitarisme}: (inspiració alemana i italiana) Es va suprimir la Constitució, les llibertats i els drets. Prohibició dels partits polítics i sindicats (excepte JONS, Falange i Central Nacional Sindicalista (sindicat vertical))

\item \textbf{Caràcter Unitari i Centralista}: Es van abolir els estatus i marginar les llengues/cultures Catalanes, Basques i Gallegues. 

\item \textbf{Repressió} sistemàtica i planificada del vençuts i oposició.

\item \textbf{Censura} rígida i utilització dels mitjans de comunicació com instruments de propaganda. 

\paragraph{Los Tres Pilares}
\begin{enumerate}
	\item \textbf{L'exèrcit} va ser el suport més destacat del règim dictatorial militarista, que va funcionar com a instrument de repressió política. Bona part dels ministres eren militars.
	
	\item \textbf{Els partits únics} eren la JONS i la Falange que es dedicaven a fabricar els missatges ideològics pel règim. 
	
	\item \textbf{L'Esglèsia} va obtenir beneficis econòmics i l'estat va a passar a ser confessional, va tindre una gran influència en el sistema educatiu i les morals/valores catòlics van ser la norma per a la societat espanyola.  
\end{enumerate}
\end{itemize}

\subsection{Autarquia}
Al voler donar al país de legalitat jurídica es van crear les \textbf{Leyes Fundamentales}:
\begin{itemize}
\item \textbf{Fuero del Trabajo}:\\
Regulació de les funcions laborals i principis del nacionalsindicalisme.

\item \textbf{Ley de las Cortes}:\\
Creació de las Cortes com òrgan legislador de caràcter consultiu i sota el control de Franco. 

\item \textbf{Fuer de los Españoles}:\\
Declaració dels drets segons el Movimiento i sense ninguna garantia de poder complir-los. 

\item \textbf{Ley del Referéndum Nacional}:\\
El cap de l'estat podria convocar un referéndum quan volgués. 

\item \textbf{Ley de Succesión}:\\
Espanya es un regne i la monarquia s'implementaria després del Franquisme. 
\end{itemize}

Els \textbf{procuradors} de les Cortes eren assignats des de dalt, era un òrgan purament assesorador i consultiu. Al definir els procuradors a partir de \textbf{quatre terços} (familiar, sindical, institucional i local), es diu que es un sistema de \textbf{democràcia orgànica}, ja que els procuradors familiars podien ser escollits pels caps de família. 

També es van organitzar \textbf{el sindicat} (CNS) de manera \textbf{vertical} que incloïa tant a les empreses com els treballadors.  

Els referèndums estaven totalment amanyats a favor del sí (els resultats estaven definits abans de que s'obresin les taules electorals). 

\pagebreak

Un dels principals objectius del Franquisma era un sistema econòmic autosuficient, \textbf{la autarquia}:
\begin{itemize}
\item \textbf{Reglamentació dels comerç exterior}:\\
Era regulat per l'estat, que volia reduir al màxim les importacions del productes de primera necessitat. Els productes importats (petroli) es van encarir molt i això va resultar en una \textbf{gran escassetat dels béns de consum}. Les matèries primes va afectar a l'energía que així mateix afectava a la producció industrial. 

\item \textbf{Foment de la indústria}:\\
Es va fundar el \textit{Instituto Nacional de Industria} que promovia una nova política industrial i lleis a favor de la creació d'empreses públiques (d'importancia estratègica com hidrocarburs, vehicles, aviació/naval i siderúrgia). 

\item \textbf{Regulació estatal de la comercialització i dels preus}:\\
Es van baixar els preus oficials dels productes de primera necessitat (cereals, llegums, vi, oli, patates...). Això va provocar una baixada en la producció que va resultat en una pobre productivitat agrícola (eficiència per metre quadrat). 
\end{itemize}

A través d'aquestes mesures es va crear una etapa d'\textbf{estancament econòmic} (col·lapse del comerç exterior, descens de la producció i el consum) i disminució dels nivell de vida de la població. 

Amb el control estatal del mercat, els agricultors es van veure obligats a vendre a un preu molt regulat. Això juntament amb la creació de les cartilles de racionament (escassesa d'aliments) van comportar a la creació d'\textbf{un mercat negre} en el qual es vendrien les mercaderies a un preu fins a quatre vegades superior al original (\textbf{estraperlo}). Afectava als aliments, matèries primes i productes industrials. Funcionava gracies a la corrupció i tolerància de moltes autoritats. 

Les condicions de vida eran molt pobres. El país va tardar fins a tres vegades més temps de recuperar-se de la guerra que altres països europeus després de la WWII. Però aquest fet també té a veure amb el Pla Marshall i com no va afectar aquest a Espanya però sí als altres països. 

La qualitat de vida es va veure reduïda com a resultat dels grans nivells d'inflació i dels salaris baixos, que sempre estaven per sota dels preus. Això va causar que les jornades de treball fossin de 12-14 hores. També va incrementar la taxa de mortalitat, sobretot l'infantil, per sota de nivells anteriors a 1935. Va augmentar la presencia d'habitatges no dignes i el barraquisme. 

\subsubsection{Nacionalcatolicisme}
A causa del malestar de la població van haver-hi dues protestes massives, la \textbf{vaga general de Bilbao} al 48 i el \textbf{moviment popular de Barcelona} al 51 a causa del augment del preu del tramvia. 

Les dificultats econòmiques van obligar a Franco a canviar els trets més feixistes del règim. Es va tirar més cap al nacionalcatolicisme que va donar més pes al catòlics en comptes del militars falangistes. Volia que es veies a Espanya com un país catòlic. 

\subsubsection{Reorientació Econòmica i Movimiento Nacional}
A meitat dels anys 50 la situació internacional era millor, però el país seguia tenint gran problemes interns. Van haver-hi una nova onada de protestes que va esclatar en un \textbf{moviment vaguístic} important, sobretot en empreses químiques i del metall. 

Llavors al 1957, Franco va fer una altre remodelació al règim que promocionaria encara més els catòlics al gobierno, els anomenats \textbf{tecnòcrates del Opus Dei}, persones que van ocupar llocs decisius en la direcció econòmica. Es va començar una liberalització econòmica del país i una apertura comercial al exterior, per tal de salvar l'estat que estava en un punt de fallida. 

De les primeres coses que es van fer, va ser la promulgació de la \textbf{Ley de Principios del Movimiento Nacional} que actualitzà el règim sense canviar els aspectes dictatorials. Els funcionaris públics havien de jurar aquest principis abans d'asumir el seu càrrec. 

La FET i la JONS volien que aquesta \textbf{Ley} atorgués grans poders al partit però degut a la resposta que van donar l'Església, militars i empresaris, el \textbf{Movimiento} es va quedar com una agrupació de famílies franquistes sense predominança falangista. 

\subsection{Relacions Internacionals}
\subsubsection{Anys de Boicot Internacional 45-47}
La postguerra mundial va significar l'aillament internacional d'Espanya, ja que no va formar part dels aliats ni dels Pla Marshall. Les recent creades Nacions Unides van condemnar explícitament el règim de Franco. 

França va tancar les fronteres i es van \textbf{retirar els ambaixadors} de Madrid. L'aillament internacional va ser presentat com una maniobra estrangera per desprestigiar el país i portar a Espanya a una nova guerra civil. 

Espanya al estar fora del Pla Marshall no va rebre cap ajuda econòmica i va ser exclosa de la OTAN que es va crear al 1949. 

\subsubsection{Guerra Freda en endavant}
Al iniciar la Guerra Freda amb els \textbf{dos blocs ideològics}, Estats Units volia comptar amb un gran aliat anti-comunista a Europa i Espanya entrava perfectament en aquest paper. 

Por a poc es va acceptar internacionalment al país, al 1947 EUA es va negar a imposar noves sancions a Espanya i va pressionar perquè l'ONU no ratifiqués la condemna dels anteriors anys. Al 1950 els ambaixadors van tornar a Madrid. 

Al 1953 va arribar el reconeixement internacional definitiu quan es van fer \textbf{acord amb la Santa Seu} i \textbf{amb els Estats Units}. El concordat amb el papat va oficialitzar la confessionalitat de l'estat i va dotar un estatus de privilegi a l'Església Catòlica. 

Amb els EUA es va pactar l'establiment i ús d'una sèrie de bases militars a la península, també el aprovisionament de material per modernitzar l'exèrcit. Franco va aconseguir el suport polític d'una gran potencia i va assegurar el reconeixement internacional del país.  

\subsection{Desarollismo}
Al final dels anys 50 les reserves del Banc d'Espanya ja estaven acabant-se, hi havia un gran dèficit comercial, una elevada inflació. Com es de costum, l'OCDE i el FMI van intervenir i a canvi de préstecs es van acceptar les típiques mesures econòmiques que demanen aquestes organitzacions. 

Es va desenvolupar el \textbf{Plan de Estabilización}, que consistia en:
\begin{enumerate}
\item \textbf{Estabilització de l'Economia}: Per reduir la inflació es va elevar les taxes d'interès\footnote{\textit{classic}}, limitar els crèdits bancaris i congelar els sous\footnote{D'aquesta manera no baixes encara més ni pugen, ja que això últim causaria que els preus siguin encara més elevats.} (reducció inflació i dèficit públic)

\item \textbf{Liberalització Interior}: Es van limitar els organismes estatals interventors i es va posar fi a la regulació de preus. 

\item \textbf{Liberalització Exterior}: Eliminació dels obstacles (aranzels) del comerç exterior i de la inversió estrangera. Per facilitar els intercanvis es va devaluar la pesseta un 50\% respecte al dòlar americà. 
\end{enumerate}
Al llarg termini aquest polítiques pretenien implementar l'economia espanyola en el mercat global i estimular el creixement econòmic. Aquest \textbf{\textit{desarrollismo} econòmic} va ser la garantia principal d'estabilitat social i de la continuïtat política, sense qüestionar els principis ideològics de la dictadura. El predomini dels tecnòcrates va fer relegar als falangistes a posicions més socials (treball, habitatge, sindicat) a canvi que els tecnòcrates estiguin en llocs com economia o industria.  

Per continuar amb les reformes es van desenvolupar els \textbf{Planes de desarollo economico y social}, planificacions orientatives que tenien l'objectiu d'impulsar l'activitat del sector públic. 

Es van crear tres plans quadriennals que eren controlats per la recent creada \textbf{Comisaria del Plan de Desarrollo}. Aquests plans tenien dues vies d'actuació:
\begin{itemize}
\item Accions Estructurals: que pretenien resoldre les deficiències industrials
\item Pols de Desenvolupament: que havien de reduir els desequilibris regionals a partir de la promoció de l'industria en zones d'escasa industrialització. 
\end{itemize}
Els \textbf{resultats de tot això van ser molt limitats} perquè hi havien pocs recursos i es van invertir ineficientment. Però si que va servir per dotar al país amb infraestructura i algunes matèries bàsiques.  

També va haver-hi una reforma d'Hisenda (\textbf{reforma fiscal}) que va impulsar la recollida d'impostos de forma indirecta, degut a això la reforma va ser insuficient. Es va intentar limitar el dèficit públic i controlar la despesa estatal.  

No obstant, els pressupostos destinats a \textbf{despeses socials i infraestructures} van créixer notablement sobretot en relació a els millors tractats tradicionalment com defensa i governació. 

\subsubsection{Anys de Creixement Econòmic}
La \textbf{diversificació industrial} va comportar un augment notable dels sectors metal·lúrgic (cotxes, electrodomèstics), del químic (farmacèutica, plàstics, detergents) i de l'alimentació. Les industries que requerien de més tecnologia van créixer degut als baixos salaris que hi havia al país respecte a altres economies europees. 

La situació industrial va quedar més o menys igual: Les grans zones tèxtils seguien (Catalunya i Basc), però també es van desenvolupar noves zones com Madrid. 

A Cat van créixer les industries metal·lúrgiques, les químiques (Tarragona) i les farmacèutiques. 

Es van crear moltes \textbf{explotacions ramaderes modernes} que van provocar un \textbf{èxode rural} cap a les urbes degut al menor requeriment de mà d'obra que va provocar l'augment de la producció/eficiència agrícola/ramadera. 

L'augment de la població urbana va crear un \textbf{augment del pes del serveis} en l'economia. També va ser influenciat per l'augment de les xarxes de comerç, millora dels transports, serveis públics i \textbf{el turisme}. 

La \textbf{construcció també va rebre un augment} degut a la necessitat de nous edificis a les ciutats i la necessitat de nova infraestructura pel turisme. 

Al necessitar finançar les noves inversions econòmics, el \textbf{sector bancari} va guanyar pes, per poder d'aquesta manera invertir en les noves indústries, atorgar crèdits... La concentració bancaria va ser molt gran: els 7 bancs més grans controlaven el 80\% dels recursos financers del país. 

Tot això va provocar una \textbf{dependència de l'economia en factors externs}. Es necessitava importar tecnologia i atraure inversions estrangeres. 

Per última va haver-hi una \textbf{manca de recursos públics} perquè no es va introduir una reforma fiscal progressiva, i l'estat no era capaç de complir amb la distribució de les rendes no impulsar adequadament la construcció de nous habitatges. 

\subsubsection{Influencia d'Europa}
Va haver-hi una \textbf{vinculació de l'economia espanyola al mercat internacional} i l'aprofitament de l'expanció econòmica europea. Els recursos financers venien principalment de: 
\begin{enumerate}
	\item Trameses dels emigrants
	\item Divises dels turistes
	\item Inversions estrangeres
\end{enumerate} 

\subsection{Oposició}


\subsection{Catalunya}
Catalunya al ser un territori particularment hostil amb els guanyadors de la guerra, la seva identitat nacional i l'anterior govern autonòmic, va estar en el punt de mira (\textbf{règim amb unes característiques espacials}). 

Al entrar les tropes a Cat es va anular \textbf{l'estatut}, posant fi al \textbf{autogovern}. Es volia formar una 'Catalunya Espanyola' i treure el catalanisme de la cultura/societat. 

El \textbf{suport al franquisme a Cat} venia dels industrials, propietaris agraris, grans comercials i burgesos. Que van identificar-se amb el nou règim. 

La militància de la FAI no era gens abundant, estava formada pels pocs falangistes, i altres grups d'extrema dreta (carlins, excombatents nacionals). 

Els \textbf{alts càrrecs eren assignats des de dalt}, i eren majoritàriament d'altres llocs d'espanya. Es volia que les autoritats controlin el territori i es desconfiava dels polítics catalans. 

Però que la major part dels càrrecs menors (ajuntaments, diputacions, Movimiento) corresponien a \textbf{franquistes catalans}. La nova classe política tenia excombatents, falangistes, antics lerrouxistes i membres de la Lliga Catalana. Eren persones que provenien de les altes elits econòmiques. 

\subsubsection{Repressió}
Van ser \textbf{executats 5000 catalans} (Lluís Companys, Joan Peiró). Es van 'jutjar' a tribunals militars a prop de 78000 persones. El nombre d'exiliats va ser de 60000 aproximadament. 

Va haver-hi una gran \textbf{depuració de treballadors} (funcionaris i d'empreses), també es van depurar els col·legis professionals. També es va posar en marxa una \textbf{gran confiscació de propietats} de les persones/entitats afectades. 

\subsubsection{Identitat Catalana}
Es va intentar \textbf{esborrar l'identitat catalana}. El \textbf{castellà} es va imposar com a única llengua llengua oficial, el català es considerava com un dialecte inapropiat que estava \textbf{prohibit en públic}. També es va considerar causa de sanció i multa quan s'usava en privat. Es va impedir la publicació de llibres, diaris, teatre, cinema en català totalment. Fins i tot es va intentar erradicar de les esglésies. 

També es van prohibir les festivitats culturals catalanes i els \textbf{símbols de la catalanitat} (banderes, himnes, cançons...). Les institucions catalanes culturals es van tancar, algunes com l'Institut d'Estudis Catalans van passar a actuar en la clandestinitat. A més a més es van treure els llibres en català de les escoles i biblioteques (també els considerats contraris a la cultura oficial). 

Aquestes polítiques tan agressives van provocar a mig termini l'efecte contrari: va haver-hi una forta resposta cultural/cívica que va relacionar \textbf{el catalanisme amb l'antifranquisme}. Es va crear una mena de resistència passiva en la població amb un activisme que volia mantenir la llengua i cultura catalanes.  
 
\subsubsection{Economia}
Durant l'etapa de creixement econòmic: 

La situació d'importancia econòmica de Catalunya anterior a la guerra va fer que creixés més que la mitjana espanyola, tenia la major distribució de renda per comunitats i la major densitat industrial, principalment impulsades per l'augment de la població degut a les immigracions interiors. També concentrava la major part del turisme espanyol (30\% de tot el turisme), principalment per la costa brava i daurada. 

\subsection{Crisi Final}

\section{La transició i la Democràcia (1975-1986)}

\pagebreak

\section{Recopilació de preguntes}

\begin{itemize}
\item Tema 5: 
\subitem a) Expliqueu \textbf{la implantació del règim franquista a Catalunya} i què significà en l’àmbit polític, repressiu, econòmic i social fins al 1945.
\subitem b) Expliqueu  \textbf{la  situació  internacional  del  règim  franquista}  entre  el  1945  i  el  1953,  fent  esment  dels  tractats  internacionals  que  signà,  així  com  de  les  lleis  que  aprovà  en  aquest  mateix període.

\item Tema 3: 
\subitem a) Expliqueu \textbf{l’adveniment de la Segona República a Espanya}, la tasca del Govern provisional, les eleccions parlamentàries del 1931 i la nova Constitució. Expliqueu igualment qui foren Niceto Alcalá Zamora, Manuel Azaña i Francesc Macià i quina importància tingueren en aquest mateix període. 
\subitem b) Expliqueu les \textbf{reformes} militar, territorial, agrària, laboral, religiosa i educativa portades a terme durant el \textbf{primer bienni republicà}.

\item Tema 4:
\subitem a) Expliqueu \textbf{l’esclat de la Guerra Civil a Catalunya} i la situació política, econòmica i social que s’hi visqué fins als Fets de Maig del 1937, sense explicar aquests fets.
\subitem b) \textbf{Expliqueu  la  batalla  de  l’Ebre}:  les  seves  causes,  el  desenvolupament  i  les  conseqüències.  Expliqueu igualment el \textbf{final de la guerra a Catalunya}, incloent-hi l’exili a França.

\noindent\rule{\linewidth}{0.4pt}

\item Tema 6:
\subitem a) Compareu els resultats de les \textbf{eleccions generals del 1979 i del 1982}. Expliqueu quins partits les guanyaren i els fets fonamentals del període comprès entre aquests dos comicis.
\subitem b) Expliqueu el \textbf{període de la transició a la democràcia a Espanya}, que inclou des de la celebració  de  les  primeres  eleccions  democràtiques  fins  a  l’aprovació  de  la  Constitució  del  1978. Expliqueu també els trets fonamentals d’aquesta constitució.

\item Tema 5:
\subitem a) Compareu les dues \textbf{polítiques econòmiques} que aplicà el règim franquista mentre va ser vigent, tant des del punt de vista de les mesures aplicades a cada etapa com dels seus resultats econòmics i socials, i expliqueu com afectaren \textbf{Catalunya}. 
\subitem b) Expliqueu la repressió \textbf{franquista a Catalunya} durant tot el període de vigència d’aquest règim.

\noindent\rule{\linewidth}{0.4pt}

\item Tema 3: 
\subitem a) Expliqueu  DUES  de  les  \textbf{reformes}  que  es  van  dur  a  terme  durant  el  \textbf{primer  bienni}  de  la  Segona República espanyola.
\subitem b) Expliqueu  el  denominat  «\textbf{bienni  negre}»:  els  esdeveniments  més  destacats  que  s’hi  van  produir i les conseqüències que van tenir.

\item Tema 4:
\subitem a) Expliqueu  \textbf{l’esclat  de  la  Guerra  Civil  a  Catalunya},  la  violència  revolucionària  que  s’hi  va  produir i què va ser el Comitè de Milícies Antifeixistes. Expliqueu també com va evolucionar la situació política, econòmica i social a Catalunya fins als Fets de Maig del 1937, sense explicar aquests fets.
\subitem b) Expliqueu els \textbf{Fets de Maig del 1937} i la \textbf{posterior evolució política i militar} de la Guerra Civil a Catalunya fins que es va completar la conquesta franquista de tot el territori català.

\item Tema 5:
\subitem a) Expliqueu  quin  \textbf{tipus  de  règim  va  ser  el  règim  franquista},  quins  suports  interiors  i  exteriors va tenir i quines lleis, institucions i partit va crear fins al 1947.
\subitem b) Expliqueu l’\textbf{oposició} al règim durant tot el franquisme. 

\noindent\rule{\linewidth}{0.4pt}

\item Tema 4:
\subitem a) Expliqueu l’evolució \textbf{política de l’Espanya republicana} durant la Guerra Civil.
\subitem b) Expliqueu la Guerra Civil des del \textbf{punt de vista militar}. Destaqueu-ne les diferents fases i expliqueu-ne també DOS dels fets bèl·lics més importants.

\item Tema 3:
\subitem a) Expliqueu  \textbf{l’adveniment  de  la  República  a  Catalunya},  la  Generalitat  provisional  i  la  seva  actuació.  Expliqueu  també  el  debat,  l’aprovació  i  les  característiques  de  l’Estatut  d’autonomia del 1932.
\subitem b) Expliqueu  les  causes  i  el  desenvolupament  dels \textbf{ Fets  d’Octubre  del  1934}  i  l’evolució  de  \textbf{l’autonomia de Catalunya} des d’aleshores fins al juliol del 1936.

\item Tema 6: 
\subitem a) Expliqueu l’evolució política d’Espanya des de la \textbf{mort de Franco fins a la celebració de les primeres eleccions democràtiques}, sense explicar-ne els resultats.
\subitem b){\tiny } Expliqueu  l’evolució  política  d’Espanya  des  de  la  \textbf{celebració  de  les  primeres  eleccions  democràtiques fins al triomf electoral del PSOE}.

\noindent\rule{\linewidth}{0.4pt}

\item Tema 5:
\subitem a) Expliqueu què va significar la \textbf{implantació del règim franquista a Catalunya} des dels punts de  vista  polític,  econòmic  i  cultural.  Expliqueu  també  la  \textbf{repressió  franquista}  i  la  lluita  antifranquista a Catalunya fins al 1959.
\subitem b) Expliqueu  el  règim  franquista  a  Catalunya  durant  \textbf{l’etapa  desarrollista}  i  els  canvis  econòmics  i  socials  que  va  comportar.  Expliqueu  també  la  \textbf{lluita  política  antifranquista  a  Catalunya} des dels anys seixanta fins a la mort del dictador.

\item Tema 3: 
\subitem a) Expliqueu  breument  qui  foren  Niceto  Alcalá  Zamora,  Manuel  Azaña,  José  María  Gil-Robles,  Alejandro  Lerroux  i  Francisco  Largo  Caballero;  seguidament,  expliqueu  \textbf{DUES  de  les  reformes  fonamentals}  que  es  portaren  a  terme  durant  el  bienni  1931-1933  i  \textbf{DUES característiques de la Constitució republicana}.
\subitem b) Expliqueu  els  resultats  de  les  eleccions  \textbf{legislatives  espanyoles  del  novembre  del  1933}  i  compareu-los  amb  els  resultats  de  les  \textbf{eleccions  del  febrer  del  1936}.  Feu  esment  dels  \textbf{governs que es van formar i de les polítiques que van aplicar en cada cas}.

\item Tema 6:
\subitem a) Expliqueu \textbf{l’evolució política de Catalunya} des de l’aprovació de l’Estatut d’autonomia \textbf{del 1979 fins al 1986}. Expliqueu també les característiques de l’Estatut.
\subitem b) Expliqueu l’evolució política de Catalunya \textbf{des de la mort de Franco fins a l’aprovació de l’Estatut d’autonomia}, sense explicar aquest darrer.

\noindent\rule{\linewidth}{0.4pt}

\item Tema 6:
\subitem a) Expliqueu  el  procés  de  \textbf{recuperació  de  l’autonomia  de  Catalunya}  fent  esment  del  retorn  del  president  Tarradellas,  de  la  Generalitat  provisional,  de  l’Estatut  del  1979  i  les  seves  característiques.
\subitem b) Expliqueu  \textbf{la  transició  cap  a  la  democràcia},  des  de  la  mort  de  Franco  fins  a  les  primeres  eleccions  democràtiques  del  15  de  juny  de  \textbf{1977},  fent  esment  dels  resultats  d’aquestes  a  Catalunya i al conjunt de l’Estat espanyol.

\item Tema 3:
\subitem a) Expliqueu les \textbf{eleccions legislatives de novembre del 1933} i l’evolució de la política espanyola des d’aquestes eleccions \textbf{fins a les de febrer del 1936}, sense explicar aquestes últimes.
\subitem b) Expliqueu  l’Estatut  d’\textbf{autonomia  de  Catalunya  del  1932}.  Compareu  la  Llei  de  reforma  agrària espanyola i la Llei de contractes de conreu catalana.

\item Tema 5:
\subitem a) Expliqueu el règim \textbf{franquista a Catalunya durant l’etapa desarrollista} i els canvis econòmics i socials que va comportar, tot relacionant-los amb l’evolució de l’oposició en aquesta mateixa etapa.
\subitem b) Expliqueu la \textbf{crisi final del franquisme} fins a la mort del dictador, incloent-hi el paper que hi va tenir l’oposició.

\noindent\rule{\linewidth}{0.4pt}

\item Tema 4:
\subitem a) Descriviu \textbf{la batalla de l’Ebre} i expliqueu què va significar i quines conseqüències va tenir. Expliqueu igualment la fi de la Guerra Civil a Catalunya.
\subitem b) Expliqueu  l’\textbf{inici  de  la  Guerra  Civil  a  Catalunya}  i  la  situació  política  i  social  que  s’hi  va  viure  fins  als  \textbf{Fets  de  Maig  del  1937}.  Expliqueu  també  les  causes  i  el  desenvolupament  d’aquests fets.

\item Tema 3: 
\subitem a) Compareu els resultats de les \textbf{tres eleccions generals celebrades} durant la Segona República fins a l’inici de la Guerra Civil espanyola.
\subitem b) Expliqueu TRES de les \textbf{reformes del primer bienni republicà}.

\item Tema 6:
\subitem a) Expliqueu  els  resultats  de  les  \textbf{eleccions  generals}  del  15  de  juny  de  \textbf{1977 a Catalunya},  el  retorn del president Tarradellas i el \textbf{restabliment de la Generalitat}, així com l’elaboració i les característiques fonamentals de l’Estatut d’autonomia de Catalunya del 1979.
\subitem b) Expliqueu les \textbf{eleccions del 1980 al Parlament de Catalunya}, els seus resultats i l’obra de govern de la Generalitat de Catalunya entre el 1980 i el 1986.

\end{itemize}


	
\end{document}