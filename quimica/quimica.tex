% !TeX spellcheck = es_ES
\documentclass[arial,a4paper,print]{article}

\usepackage{amsmath}
\usepackage{helvet}
\usepackage{lipsum}
\usepackage{multirow}
\usepackage{array}
\usepackage{physics}
\usepackage[version=4]{mhchem}
\usepackage{epsfig}
\usepackage{amssymb}
%\usepackage{svrsymbols}
\usepackage{siunitx}
\usepackage{graphicx}
\usepackage{subcaption}
\usepackage[labelfont=sc, font={footnotesize, singlespacing}]{caption}
\usepackage[margin=2cm]{geometry}  

\renewcommand{\familydefault}{\sfdefault}

\usepackage[spanish]{babel}
%opening
\title{Química: Selectividad 2022}
\author{tomiock}

\begin{document}
	
	\maketitle
	
	\section{La radiación, los átomos y las moleculas}
	\subsection{Radiación EM y interaciones entre átomos}
	La radiación electromagnética se puede caracterizar mediante diversas magnitudes: 
	\begin{enumerate}
		\item Longitud de onda ($\lambda$): \\
		Distancia mínima entre dos puntos en fase (mismo estado de vibración. 
		\item Periodo ($T$): \\
		El tiempo que tarda la onda en recorrer la longitud de onda. 
		\item Frecuencia ($\nu$): \\
		El número de longitudes de onda que pasan por un punto determinado en un segundo. Se puede relacionar con su longitud de onda y velocidad de propagación: 
		\begin{equation*}
			\nu = \frac{c}{\lambda}
		\end{equation*}
		En el caso de la radiación EM su velocidad de propagación es $c$, la velocidad de la luz, claro. 
	\end{enumerate}
	
	Se relacionan la energia con la lontitud de onda y la frecuencia de la siguiente manera:
	\begin{equation*}
		E = h\frac{c}{\lambda} 
	\end{equation*}
	
	\subsubsection{Radiación Infraroja}
	Los fotones ubicados en el espectro infrarojo no tienen suficiente energia para poder provocar la transición electrónica de los átomos, pero si que los hacen vibrar. Cuando la radiación infraroja es absorbida por los átomos de gases en las atmósferas, les hacen vibrar a major frecuencia y por lo tanto augmentan la temperatura. 
	
\end{document}